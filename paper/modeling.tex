\section{RD-Modeling} \label{sec:modeling}

The target of our model is to predict the objective video quality which we choose VMAF as our main quality metric.
VMAF is a quality metric developed by Netflix.
It predicted the video quality by existing image metrics and some other features, such as Visual Information Fidelity (VIF), Detail Loss Metric (DLM), and Mean Co-Located Pixel Difference (MCPD).

We vary the following inputs of the model in our experiments:
\begin{itemize}
\item {\bf QP} trades off the bitrate and RGBD video quality.
MPEG group suggests using a smaller QP for depth video (80\% of the corresponding RGB videos) because depth imposes significant impacts on synthesized quality~\cite{jung2020common}.
We set (RGB video) QP $\in$ \{20, 36, 44, 48, 50\}, targeting 5 to 50 Mbps total bitrates.
\item {\bf Target view trajectory} includes: (i) yaw, roll, and pitch, which specify the user {\em orientation} and (ii) surge, heave, and sway, which specify the user {\em position}.
\end{itemize}